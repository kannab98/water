
\documentclass[final,hyperref={pdfpagelabels=false},16pt]{beamer}
\usetheme{Berlin}
\usefonttheme{professionalfonts} % using non standard fonts for beamer
\usefonttheme{serif} % default family is serif

\usepackage[russian]{babel}
\usepackage[utf8]{inputenc}
\usepackage[orientation=portrait,size=a0,scale=1,debug]{beamerposter}
\usepackage
	{
		% Дополнения Американского математического общества (AMS)
		amssymb,
		amsfonts,
		amsmath,
		amsthm,
		physics,
		graphicx
		}
\setbeamertemplate{navigation symbols}{} % минус навигация
\let\Tiny=\tiny % решает проблему со шрифтами в TexLive
\setbeamertemplate{headline}{
	\leavevmode
	\begin{beamercolorbox}[wd=\paperwidth]{headline}
		\vspace{2ex}\\
		\begin{columns}[c]
			\begin{column}{.09\paperwidth}
			\end{column}
			\begin{column}{.675\paperwidth}
				\raggedleft
				\usebeamercolor{title in headline}{\textbf{\LARGE{\inserttitle}}\\[1ex]}
				\usebeamercolor{author in headline}{\large{\insertauthor}\\[1ex]}
				\usebeamercolor{institute in headline}{\small{\insertinstitute}\\[1ex]}
			\end{column}
			\begin{column}{.25\paperwidth}
				\begin{center}
                    \includegraphics[width=.8\textwidth, height=.39\textwidth, keepaspectratio]{rf3}
				\end{center}
			\end{column}
			\begin{column}{.03\paperwidth}
			\end{column}
		\end{columns}
		\vspace{2ex}\\
	\end{beamercolorbox}

 	\begin{beamercolorbox}[wd=\paperwidth]{lower separation line head}
		\rule{0pt}{2pt}
	\end{beamercolorbox}
	\vskip-2cm
	}

\setbeamertemplate{footline}{
	\begin{beamercolorbox}[wd=\paperwidth]{upper separation line foot}
		\rule{0pt}{2pt}
	\end{beamercolorbox}
	\leavevmode%
	\begin{beamercolorbox}[ht=4ex,leftskip=1cm,rightskip=1cm]{author in head/foot}%
		{Оформлено при помощи издательской системы \LaTeX}
        \hfill
		\today
		\hfill
		\href{mailto:my_address@wikibooks.org}{github.com/kannab98}
		\vskip1ex
	\end{beamercolorbox}
	\vskip0pt%
	\begin{beamercolorbox}[wd=\paperwidth]{lower separation line foot}
		\rule{0pt}{2pt}
	\end{beamercolorbox}
	}

\title{Численное моделирование морской поверхности}
\author{Понур К.А.$^{1}$, Караев В.Ю.$^2$, Рябкова М.С.$^2$}
\institute{$^{1}$ Нижегородский Государственный Университет им. М.Ю. Лобачевского \\
          $^2$ Институт Прикладной Физики Российской Академии Наук}
\date{\today}
\newcommand{\tM}{\widetilde{M}}
\usepackage{mathtools}
\mathtoolsset{showonlyrefs}
\begin{document}
  \begin{frame}[t]{} 
    \begin{columns}[t]
      \begin{column}{.48\linewidth}
        \begin{block}{Введение}
            Для изучения и мониторинга состояния морской поверхности активно используются радиолокаторы. 
            Благодаря орбитальным скаттерометрам измеряется поле приводного ветра в Мировом океане, радиовысотомеры измеряют высоту значительного волнения и средний уровень
            морской поверхности. Для обнаружения разливов нефти и нефтепродуктов активно используются радиолокаторы с синтезированной апертурой.

            Современные достижения в радиолокационном зондировании базируются на результатах исследования рассеяния электромагнитных волн морской поверхностью. Тем не менее, остается ряд нерешенных 
            вопросов по моделям рассеяния. Также, существующие радиолокационные системы не всегда позволяют получить необходимую информацию о состоянии приповерхностного слоя океана, что обуславливает необходимость совершенствования измерительной аппаратуры. 

            На этапе поиска оптимальной схемы измерения необходимо оценить вклад в спектральные и энергетические характеристики отраженного сигнала параметров волнения, течений, поверхностных пленок. Численное моделирование является мощным инструментом, позволяющим решать эти задачи.
        \end{block}
        \begin{block}{Моделирование}
            Рассмотрим задачу моделирования морской поверхности по заданному спектру волнения.

            Спектр двумерного волнения представим в виде функции с разделяющимися переменными (см. рис. \ref{fig:spec}):
        \begin{equation}
            \label{eq:}
            S(\vec k) = S(k) \Phi_k(\phi), ~ k = \sqrt{k_x^2+k_y^2},~ \phi=\arctg{\frac{k_x}{k_y}},\quad \int\limits_{-\pi}^{\pi} \Phi_k(\phi) \dd{\phi} = 1.
        \end{equation}
        
            Поверхность представим как сумму гармоник с детерменированными амплитудами и случайными фазами
        \begin{equation}
            \zeta(\vec r, t)= \sum\limits_{n=1}^N \sum_{m=1}^M A_n(k_n)\cdot 
            \Phi_{k_nm}(\phi_m) \cos(\omega_n t + \vec k_n \vec r + \psi_{nm}),
        \end{equation}
        Амплитуда, которая является мощностью на интервале $\Delta k_n$, вычисляется по спектру моделируемой поверхности
        \begin{equation}
            A_n(k_n)=\sqrt{ \int\limits_{(\Delta k_n)}  2 S(k) \dd{k}}
        \end{equation}
        $\Phi_{nm}$ -- азимутальное распределение, вычисляемое следующим образом:
        \begin{equation}
        \Phi_{nm}(k_n,\phi_m)=\sqrt{\Phi(k_n,\phi_m) \Delta \phi},
        \end{equation}
        $\Delta \phi$ -- шаг по углу
        \begin{figure}[h]
            \begin{minipage}{0.24\linewidth}
                \includegraphics[width=\linewidth]{example-image-a}
                \centering
                (a)
            \end{minipage}
            \begin{minipage}{0.24\linewidth}
                \includegraphics[width=\linewidth]{example-image-a}
                \centering
                (b)
            \end{minipage}
            \begin{minipage}{0.24\linewidth}
                \includegraphics[width=\linewidth]{fig/full_angles1}
                \centering
                (c)
            \end{minipage}
            \begin{minipage}{0.24\linewidth}
                \includegraphics[width=\linewidth]{fig/full_angles2}
                \centering
                (d)
            \end{minipage}
            \label{fig:spec}
            \caption{(a) Спектр высот $S(k)$ при фиксированном значении  $\tilde x = 20170$ и меняющейся скорости ветра 
            (b) Спектр высот $S(k)$ при фиксированном значении скорости ветра $U_{10} = 10 \frac{\text{м}}{\text{с}}$ и меняющемся разгоне 
        (c,d) Угловое распределение  $\Phi_k(\phi)$ в полярных координатах для разных значений соотношений  $\frac{k}{k_m}$, где $k_m$ - координата пика спектра $S(k)$ при фиксированной скорости ветра}
        \end{figure}
        \begin{figure}[h]
            \begin{minipage}{0.45\linewidth}
                \includegraphics[width=\linewidth]{water5}
                \centering
                (a)
            \end{minipage}
            \begin{minipage}{0.45\linewidth}
                \includegraphics[width=\linewidth]{water15}
                \centering
                (b)
            \end{minipage}
            \caption{Пример смоделированной поверхности (a) $U_{10} = 5 \text{ м/с }$ (b) $U_{10} = 15 \text{ м/с}$}
        \end{figure}
        \end{block}
        \begin{block}{Модель заостренной волны (Choppy wave model)}
            Модель заключается в нелинейном преобразовании координат
            \begin{gather}
                x = x_{0} - \sum\limits_{j} \frac{\vec k_j}{\abs{\vec k_j}} \cdot \vec x_{0} \sin(\vec k_j \vec r_{0} - \omega_j t + \phi_j) \\
                y = y_{0} - \sum\limits_{j} \frac{\vec k_j}{\abs{\vec k_j}} \cdot \vec y_{0} \sin(\vec k_j \vec r_{0} - \omega_j t + \phi_j) \\
                z =  \sum\limits_{j=1} a_j \cos(k_j \cdot \vec r_{0} - \omega_j + \phi_j)
            \end{gather}
            Или человеческим языком:
            \begin{equation}
                \label{eq:}
                \qty{\vec r, h(\vec r,t)} \to \qty{\vec r + \vec D(\vec r,t), h(\vec r,t)},
            \end{equation}
            где $\vec r = (x,y)$ -- горизонтальная координата, $D(\vec r,t)$ -- Riesz Transform
            Характеристическая функция такого процесса
            \begin{equation}
                \label{eq:}
                \theta(u) = \qty(1 - i u \sigma_1^2 + u^2 \Sigma_1) \exp{-\frac{1}{2} u^2 \sigma_0^2}
            \end{equation}
            \begin{equation}
                \label{eq:}
                \sigma^2_{\alpha \beta \gamma} = \iint \frac{ \abs{k_x}^{\alpha} \abs{k_y}^{\beta}}{\abs{\vec k}^{\gamma}} \dd{\vec k}
            \end{equation}
              \begin{figure}[h]
                    \centering
                    \begin{minipage}{0.45\linewidth}
                        \includegraphics[width=\linewidth]{example-image-a}
                        \centering 
                        (a)
                    \end{minipage}
                    \begin{minipage}{0.45\linewidth}
                        \includegraphics[width=\linewidth]{example-image-a}
                        \centering
                        (b)
                    \end{minipage}
                    \caption{Эволюция (a) <<линейной>>  и (b) <<нелинейной>>  систем, $\Delta t = 0.1 \text{ с}$}
                    \label{fig:}
                \end{figure}


        \end{block}
        
      \end{column}
      \begin{column}{.48\linewidth}
        \begin{block}{Метод <<отбеливания>> спектра}
            Предположим, что гармонические составляющие при больших $\rho$ складываются <<некогерентным>> образом. То есть  мощность шума определяется как
            \begin{equation}
                \sigma^2_{\text{шум}}= \sum_{i=1}^N \frac{b_i^2}{2}
            \end{equation}
            В области малых $\rho$ гармоники суммируются <<когерентно>> и соответствующая мощность равна
            \begin{equation}
                \tM^2(0)=\qty(\sum_{i=1}^N b_i)^2
            \end{equation}
            Введем функцию, характеризующую относительную мощность шумов
            \begin{equation}
                \label{eq:Q}
                Q = \frac{\sigma^2_{\text{шум}}}{\tM^2(0)}
            \end{equation}
            Минимизируем величину \eqref{eq:Q}, решая систему уравнений
            \begin{equation}
                \label{eq:cases}
                \pdv{Q}{b_i}=\frac{b_i}{\qty (\sum\limits_{i=1}^{N} b_i)^2} - \frac{\sum\limits_{i=1}^{N} b_i^2}{\qty (\sum\limits_{i=1}^{N} b_i)^3}, ~ i = 1\hdots N.
            \end{equation}
            Она сводится к следующей системе:  $b_i \sum\limits_{i=1}^{N} b_i -\sum\limits_{i=1}^{N} b_i^2=0 $
             \vfill
            Частным результатом решения является $b_1=b_2=\hdots=b_N$.
            \vfill
    % Задача сводится к разбиению области определения спектра на участки $\Delta k_i$, интегралы по которым от функции $S(k)$ имеют одинаковые значения.

           \begin{equation}
                \text{Для высот:}\quad 	b_i=b_1= \frac{M(0)}{N}=\frac1N \int\limits_0^{\infty} S(k) \dd{k}
           \end{equation}
           \begin{equation}
                \text{Для наклонов:}\quad 	b^{\theta}_i=b^{\theta}_1= \frac{M^{\theta}(0)}{N}=\frac1N \int\limits_0^{\infty} k^2S(k) \dd{k}
           \end{equation}
            Потребуем сопряжения в нуле всех производных  функций
            $\tM(\rho)$ и $M(\rho)$. 
            Для функции корреляции стационарной случайной функции $M(\rho)$ справедливо
            \begin{equation}
                M'_{\rho}=\pdv[2]{M(\rho)}{\rho}=\int\limits_{0}^{\infty} k^2 S(k)\cos(k \rho)\dd{k}
            \end{equation}
            А значит можно переписать наше требование в виде
            \begin{equation}
                \sum_{i=1}^N b_i k_i^{2p}=\int\limits_{0}^{\infty} k^{2p}S(k)\dd{k}, p = 1,2,\dots,N.
            \end{equation}
            Решать такую систему довольно сложно, поэтому потребуем выполнение более простого равенства
            \begin{equation}
                \sum_{i=1}^N b_ik_i^{2}=\int\limits_{0}^{\infty} k^{2}S(k)\dd{k}
            \end{equation}
            \begin{minipage}{0.49\linewidth}
                \centering Для наклонов:
                \begin{equation}
                    k_i=\sqrt\frac{N}{{\int\limits_{0}^{\infty} k^2 S(k) \dd{k}}}\cdot {\int\limits_{\Delta k_i} k^4 S(k) \dd{k}}
                \end{equation}
                \end{minipage}
                \hfill
                \begin{minipage}{0.49\linewidth}
                \centering Для высот:
                \begin{equation}
                    k_i=\sqrt\frac{N}{{\int\limits_{0}^{\infty} S(k) \dd{k}}}\cdot \int\limits_{\Delta k_i} k^2 S(k) \dd{k}
                \end{equation}
                \end{minipage}
                \vfill
                Решением будем считать суперпозицию решения системы уравнений
                \eqref{eq:cases} для высот и наклонов
                \begin{figure}[H]
                    \begin{minipage}{0.49\linewidth}
                            \centering
                            \includegraphics[width=\linewidth]{fig/split_angles}	
                    \end{minipage}
                    \hfill
                    \begin{minipage}{0.49\linewidth}
                            \centering
                            \includegraphics[width=\linewidth]{fig/split_height}
                    \end{minipage}
                    \caption{Расположении узлов по методу <<отбеливания>> спектра  для наклонов и высот соответственно. $U=10 \frac{\text{м}}{c}$, $N=25$}
                    \label{fig:splits}		
                \end{figure}

                \begin{figure}[h]
                    \centering
                    \begin{minipage}{0.49\linewidth}
                            \centering
                            \includegraphics[width=\linewidth]{example-image-a}

                            (a)
                    \end{minipage}
                    \begin{minipage}{0.49\linewidth}
                            \centering
                            \includegraphics[width=\linewidth]{example-image-a}

                            (b)
                    \end{minipage}
                    \caption{Корреляционные функции высот (a) и уклонов (b) при различных расположениях узлов. $U=10\frac{\text{м}}{\text{с}},~ N = 256$ }
                    \label{fig:}
                \end{figure}
        \end{block}
        \begin{block}{Заключение}
            Lorem ipsum dolor sit amet, consetetur sadipscing elitr, sed diam nonumy eirmod
            tempor invidunt ut labore et dolore magna aliquyam erat, sed diam voluptua. At
            vero eos et accusam et justo duo dolores et ea rebum. Stet clita kasd gubergren
            no sea takimata sanctus est Lorem ipsum dolor sit amet.
        \end{block}
        \begin{block}{Литература}
            \footnotesize
            \begin{thebibliography}{}
                \bibitem{Rytov} \textit{С.М. Рытов}, Введение в статистическую радиофизику // Изд. 2-е, перераб. и доп. - Москва : Наука, 1976. - Ч. 1. Случайные процессы \S\S 14-18, 38-42 
                \bibitem{Karaev1} \textit{В.Ю.Караев, М.Б. Каневский, Г.Н. Баландина}, Численное моделирование поверхностного волнения и дистанционное зондирование // Препринт №552 ИПФ РАН, 2002, С.1-10.
                \bibitem{Veber} \textit{В.Л. Вебер}, О моделировании случайного профиля морской поверхности // Изв. вузов. Радиофизика. 2017. Т. 60, № 4. С. 346.
                \bibitem{Karaev2} \textit{В.Ю.Караев, Г.Н. Баландина} Модифицированный спектр волнения и дистанционное зондирование // Исследование Земли из космоса, 2000, N5, C.1-12.
                \bibitem{Longe} \textit{М.С. Лонге-Хиггинс} Статистический анализ случайной движущейся поверхности // в кн.: Ветровые волны, М.: Иностранная наука, 1962, С.112-230
            \end{thebibliography}
        \end{block}
      \end{column}
    \end{columns}
  \end{frame}
\end{document}
