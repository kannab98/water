% Тип документа
\documentclass[a4paper,14pt]{extarticle}
\usepackage[utf8]{inputenc}
\usepackage[russian]{babel}
\usepackage[T2A]{fontenc}
\usepackage
    { % Дополнения Американского математического общества (AMS)
        amssymb,
        amsfonts,
        amsmath,
        amsthm,
        % Пакет для физических текстов
        physics,
        color,
        ulem,
        esint,
        esdiff,
        % 
    } 
    
\usepackage{mathtools}
\mathtoolsset{showonlyrefs=true} 

\usepackage{xcolor}
\usepackage{hyperref}
 % Цвета для гиперссылок
\definecolor{linkcolor}{HTML}{000000} % цвет ссылок
\definecolor{urlcolor}{HTML}{799B03} % цвет гиперссылок
 
\hypersetup{linkcolor=linkcolor,urlcolor=urlcolor,colorlinks=true}
\hypersetup{citecolor=linkcolor}
\hypersetup{pageanchor=false}

% Увеличенный межстрочный интервал, французские пробелы
\linespread{1.3} 
\frenchspacing 

\newcommand{\mean}[1]{\langle#1\rangle}
\newcommand*{\const}{const}
\renewcommand*{\arctg}{arctg}
%\renewcommand*{\kappa}{\kappa}
\renewcommand*{\phi}{\varphi}

\newcommand{\tK}{\widetilde K}
\renewcommand{\qty}{ }


\begin{document}
\section{Статья}%
\label{sec:stat_ia}

Запишем мощность импульса в зависимости от времени как свертку трех функций
\begin{equation}
    \label{eq:1}
    W(t) = FSSR(t) * PTR(t) * PDF(t), \text{ где} 
\end{equation}
$FSST(t)$ -- отклик плоской морской поверхности (Flat Sea Surface Response),
$PTR(t)$ - отклик радиолокатора (Point Target Response)
$PDF(t)$ -- распределение высот морской поверхности (Probability Density
Function).

В дальнейшем мы будем полагать $PDF(t)$ гауссовой функцией.

Теоретический отклик радиолокатора можем записать как
 \begin{equation}
    \label{eq:2}
    PTR(t) = \abs{\frac{\sin(\pi B t)}{\pi B t}}^2, \text{ где}
\end{equation}
$B$ -- полоса приема альтиметра.
Для того, чтобы удобнее выполнять операцию свертки, аппроксимируем отклик
радиолокатора гауссовой функцией
\begin{equation}
    \label{eq:PTR}
    PTR(t) \approx  \exp(-\frac{t^2}{2 \sigma_p^2})
\end{equation}
Согласно статье \cite{cite:PTR} можно связать дисперсию $\sigma_p$ в \eqref{eq:PTR} c
временным разрешением альтиметра $r_t$:  
\begin{equation}
    \label{eq:sigmap}
    \sigma_p = \frac{1}{2 \sqrt{2 \ln 2}} r_t
\end{equation}

Согласно работе Брауна  \cite{cite:brown}, мы можем выразить $FSSR$




\begin{thebibliography}{}
    \bibitem{cite:1} \textit{М.С. Лонге-Хиггинс}, Статистический анализ случайно
    движущейся поверхности // в книге Ветровые волны, Москва: Иностранная
    литература, 1962, стр. 112-230.
    \bibitem{cite:2} Статья по моделированию синусоидами 
    \bibitem{cite:3} \textit{В. Караев, М. Каневский, Г. Баландина}, Численное
    моделирование поверхностного волнения и дистанционное зондирование, 2000,
    Препринт № 552, Нижний Новгород, изд. ИПФ РАН, 25 стр. 
    \bibitem{cite:4} Дисперсионное уравнение
    \bibitem{cite:5} \textit{В.И. Тихонов}, Статистическая радиотехника. // 2-е
    изд., перераб. и доп. -- Москва: Радио и связь, 1982, стр. 119.
    \bibitem{cite:6} Спектр Рябковой
    \bibitem{cite:7} \textit{Гнеденко Б.В.}, Курс теории вероятностей: Учебник

    \bibitem{cite:10} \textit{В.И. Тихонов}, Статистическая радиотехника. // 2-е
    изд., перераб. и доп. -- Москва: Радио и связь, 1982, стр. 293.
    для университетов. -- 6-е изд.  -- М.: Наука, 1988. -- \S 16 
    стр. 400.
    \bibitem{cite:12} \textit{Lee-Lueng Fu, Anby Cazenave}, Satellite altimetry
    and earth sciences. A handbook of teckniques and applications, 2001,
    Academic Press, 464 p.
    \bibitem{cite:13} \textit{В. Пустовойтенко, А. Запевалов}, Оперативная
    океанография: современное состояние, перспективы и проблемы спутниковой
    альтиметрии, 2012, Севастополь, 218 с.
    \end{thebibliography}

\end{document}
